\newpage
\section{Chapter 1: Introduction}

\subsection{Promise and challenges in stem cell  therapy for neurological diseases}

Stem cells  are immature cells with the capacity to self-renew and differentiate into  multiple specialized neural phenotypes and can be used to replace lost or  damaged neurons in the central nervous system. The therapeutic potential of stem cells  holds great promises by the direct action on underlying cause of problems in  neurological diseases characterized by cell loss or damage. A number of clinical trials of stem cell therapies for strokes  and Parkinson's disease report that transplantation can lead to symptomatic  relief in patients. Nevertheless, improvements in patient outcomes remain  highly variable, and it is challenging to pinpoint which factors lead to  failure or success. Meanwhile, the snake oil industry that purports ‘miracle’  stem cells offers high hopes and breeds confusion and distrust at this early  stage of development of the therapeutic strategies, and the presumed mechanisms  responsible for graft-induced effects often remain poorly validated. A better  understanding and control of the regenerative processes subsequent to  transplantation are of crucial importance to ensure the safety and efficacy of  stem cell therapies for clinical applications. However, there is currently no  easy way to observe grafted stem cells and to monitor their developmental  process, not to mention whether they even survive in host tissue, crippling any  attempt to develop more reliable procedures.

\subsection

Background in neurological stem cell  researchTransplant of stem cells and follow-up assessmentSuccessful  engraftment of a transplant is measured by whether stem cell-derived neurons  integrate with host neurons and functionally mature into physiologically active  cells. The regenerative processes involve complex dynamics of cellular  interactions, and donor cells must be able to survive, migrate to target sites,  differentiate into specific neural phenotypes, and establish appropriate  connectivity with host neural and synaptic networks. Conventional studies on  the anatomical and the functional integration of transplanted cells commonly  rely on immunohistochemical methods combined with rabies tracing and electron  microscopy, and electrophysiological and optical recording, respectively.  Remarkably,  grafted interneuron progenitor cells have been reported to propagate from the  original site, differentiate into various mature interneurons, and integrate  into neural circuits of the postnatal and adult brains. After the  transplantation, immunohistochemistry and slice electrophysiology have been  performed on brain sections to quantify and validate the survival,  differentiation, migration, and integration of MGE-derived interneurons.  Previous studies in mice have reported that the transplanted cells exhibit a  survival rate of 20 % up to a year after transplantation (Zipancic et al.,  2010; Hunt et al., 2012; Alvarex-Dolado et al., 2006), migrate up to 3 mm from  the injection site in host cortical tissue, and differentiate into mature  interneurons with almost half of the population expressing somatostatin and  over one-quarter expressing parvalbumin (Sebe et al., 2014; Alvarex-Dolado et  al., 2006; Southwell et al., 2010). In addition, whole-cell recordings of the  transplanted cells exhibited electrophysiological properties of mature  interneurons (Hunt et al., 2013), and the inhibitory postsynaptic currents  (IPSC) showed significant increase in frequency and amplitude on pyramidal  neurons, indicating the direct synaptic connections between the progenitor  cells and the local cells (Zipancic et al., 2010; Southwell et al., 2010).  Much of these  assessments, however, heavily depend on temporally static analyses of terminal  samples averaged from different cohorts of animals at distinct time points.  Such experimental procedures are laborious, inefficient, and often time and  resource consuming. Thus, technical advances in the tools to track transplanted  cells in vivo will be imperative to perform longitudinal studies and will shed  light on the fate of transplanted cells over multiple spatial and temporal  scales which is otherwise inaccessible from snapshots of events deduced from  post-mortem tissue.  Assessing stem cell function by continuous measurement of  network activityCharacteristics  of spontaneous activity change progressively during cortical development. Early-form  neural activity is known to play a key role in proper cortical development  through activity-developmental processes. The propagation patterns and frequencies  of calcium wave carry important information to regulate neural proliferation,  differentiation, migration and connectivity.  Initially,  the network activity is dominated by a highly recurrent pattern of synchronous  network activity referred to as“network bursts.” This short burst is 0.5 - 2  seconds in duration and 0.1 - 0.5 Hz in firing rate and is followed by phases  of a silent network recovery and a low-level activity before the next  initiation. Gradually, the network activity desynchronizes and matures into  more spare and decorrelated activity, thus evolving to a network state with  better information processing capacities.  Immature  cells during development exhibit transient heightened state of plasticity, thus  increasing excitability and mediating the influx of calcium through  voltage-gated calcium channels or NMDA receptors. The collective form of  intercellular communication has been suggested to form through gap junction as  an early form preceding electrical connections via synapses, in addition to  release of signaling molecules from these cells (Malmersjo et al., 2013;  Jaderstad et al., 2010). Previous studies have shown that neural stem cells  form functional gap junctions with cells in organotypic culture as early as 2 –  18 hours after grafting.  During the  early phase, synapse numbers significantly increase leading to massive  overproduction of both excitatory and inhibitory synapses. The overproduction results  in unstructured network connectivity, which may explain the correlation of  prolonged burst duration with increased synapse density. In the subsequent  phase, this number declines with synaptic pruning to accommodate the balance between  excitatory and inhibitory synapses. Both firing rate and burst duration  decrease, and network bursts exhibit diversification of activity with richer  patterns. Yet in isolated development, the characteristic synchronized burst  activity patterns that normally extend only a few days in vivo persist  throughout the lifetime in cultured neurons, which may suggest an arrest in  development. Similar persistence of bursting was also observed in human  organoid implant in mouse cortex.  The evolution  of the activity pattern during normal cortical development can be followed  using a variety of methods. Periodic electrophysiological recording using  multi-electrode arrays (MEA) and acute optical recording of intracellular  calcium transients provide a convenient method to record simultaneous activity  in multiple neurons in vitro and ex vivo. Yet, the progression of observed  synchronized burst patterns activity behavior as implanted stem cell integrate  with a host has never been observed. And whether transplants exhibit  comparative developmental changes during maturation and integration and whether  such developmental characteristics are important for integration remain to be  determined. This feature may yield a new biomarker for evaluating functional integration  of transplanted stem cells in vivo.  electrophysiology: extracellular, patch-clamp, gridfluorescence microscopy: slice, immuno-fluorescent probes/dyes, etc.

