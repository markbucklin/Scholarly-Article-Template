\newpage
\section{Chapter 1: Introduction}

\subsection{Promise and challenges in stem cell  therapy for neurological diseases}

Stem cells  are immature cells with the capacity to self-renew and differentiate into  multiple specialized neural phenotypes and can be used to replace lost or  damaged neurons in the central nervous system. The therapeutic potential of stem cells  holds great promises by the direct action on underlying cause of problems in  neurological diseases characterized by cell loss or damage. A number of clinical trials of stem cell therapies for strokes  and Parkinson's disease report that transplantation can lead to symptomatic  relief in patients. Nevertheless, improvements in patient outcomes remain  highly variable, and it is challenging to pinpoint which factors lead to  failure or success. Meanwhile, the snake oil industry that purports ‘miracle’  stem cells offers high hopes and breeds confusion and distrust at this early  stage of development of the therapeutic strategies, and the presumed mechanisms  responsible for graft-induced effects often remain poorly validated. A better  understanding and control of the regenerative processes subsequent to  transplantation are of crucial importance to ensure the safety and efficacy of  stem cell therapies for clinical applications. However, there is currently no  easy way to observe grafted stem cells and to monitor their developmental  process, not to mention whether they even survive in host tissue, crippling any  attempt to develop more reliable procedures.

\subse

